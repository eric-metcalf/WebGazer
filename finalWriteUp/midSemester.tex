%%%%%%%%%%%%%%%%%%%%%%%%%%%%%%%%%%%%%%%%%%%%%%%%%%%%%%%%%%%%%%%%%%%%%%%%%%%%%%%%%%%%%%%%%%%%%%%%
%
% CSCI 1430 Project Progress Report Template
%
% This is a LaTeX document. LaTeX is a markup language for producing documents.
% Your task is to answer the questions by filling out this document, then to 
% compile this into a PDF document. 
% You will then upload this PDF to `Gradescope' - the grading system that we will use. 
% Instructions for upload will follow soon.
%
% 
% TO COMPILE:
% > pdflatex thisfile.tex
%
% If you do not have LaTeX and need a LaTeX distribution:
% - Departmental machines have one installed.
% - Personal laptops (all common OS): http://www.latex-project.org/get/
%
% If you need help with LaTeX, come to office hours. Or, there is plenty of help online:
% https://en.wikibooks.org/wiki/LaTeX
%
% Good luck!
% James and the 1430 staff
%
%%%%%%%%%%%%%%%%%%%%%%%%%%%%%%%%%%%%%%%%%%%%%%%%%%%%%%%%%%%%%%%%%%%%%%%%%%%%%%%%%%%%%%%%%%%%%%%%
%
% How to include two graphics on the same line:
% 
% \includegraphics[width=0.49\linewidth]{yourgraphic1.png}
% \includegraphics[width=0.49\linewidth]{yourgraphic2.png}
%
% How to include equations:
%
% \begin{equation}
% y = mx+c
% \end{equation}
% 
%%%%%%%%%%%%%%%%%%%%%%%%%%%%%%%%%%%%%%%%%%%%%%%%%%%%%%%%%%%%%%%%%%%%%%%%%%%%%%%%%%%%%%%%%%%%%%%%

\documentclass[11pt]{article}

\usepackage[english]{babel}
\usepackage[utf8]{inputenc}
\usepackage[colorlinks = true,
            linkcolor = blue,
            urlcolor  = blue]{hyperref}
\usepackage[a4paper,margin=1.5in]{geometry}
\usepackage{stackengine,graphicx}
\usepackage{fancyhdr}
\setlength{\headheight}{15pt}
\usepackage{microtype}
\usepackage{times}
\usepackage{booktabs}

% From https://ctan.org/pkg/matlab-prettifier
\usepackage[numbered,framed]{matlab-prettifier}

\frenchspacing
\setlength{\parindent}{0cm} % Default is 15pt.
\setlength{\parskip}{0.3cm plus1mm minus1mm}

\pagestyle{fancy}
\fancyhf{}
\lhead{Project Progress Report}
\rhead{CSCI 1430}
\rfoot{\thepage}

\date{}

\title{\vspace{-1cm}Project Progress Report}


\begin{document}
\maketitle
\vspace{-3cm}
\thispagestyle{fancy}

\section*{Instructions}
\begin{itemize}
  \item What are the skills of the team members?\\
  (R)yan - Javascript and node js, Juggling, Getting code to run, \\
  (E)ric - 
    understand database management issues, good understanding of how to make things real time, algorithmic thinker, high level idea generation, thinking about ethical implications of ideas\\
  (C)ecilia - graphics \\
  (K)atie - in Deep Learning, juggling \\
  
  \item What is your idea?
  \item What data will you use?
  \item What software will you use?
  \item Who will do what?
  \item What progress have you made so far?
  \item What problems to you foresee or have?
  \item Is there anything that we can do to help?
  \item 2 pages max.
\end{itemize}

\section*{Team name: \emph{RECK-it-Gazer}}

\section*{Running the webgazer demo:}


\begin{enumerate}
\item
Install node/NPM:
    https://nodejs.org/en/
\item
In command line, run: 
    npm install http-server -g
\item
Navigate to webgazer directory and run: 
    serve-https
\end{enumerate}

\begin{verbatim}
Eric's Notes:
Board notes:
    What we have?
    code 

Notes on other systems to tryout:
    xlabs - a company that is using face position 
    and eye location to get eye position
    Camgaze.js - this is a tracker that just predicts 
    OpenGazer [34]  based on open cv
    
    
\end{verbatim}

Todo over Break: 
\begin{enumerate}
\item smoothing the eye tracking position (Ryan)
\item chrome extension(?) (Ryan)
\item improve blink detector (Katie)
\item include face angle(?) (Katie)
\item Creating a model that would go from face/eyes $\rightarrow$ screen position (Eric)
\begin{enumerate}
    \item What are the models we should use?
    \item How do we combine the metadata with the actual images?
    \item If we are using deep neural networks, what is the structure of the networks we want to use?
    \item Get it running on a gpu
    \item What gpu should eric buy for his desktop?
    
\end{enumerate}
\item Create 3d version of face (to hypothetically put into machine learning algorithm)
\end{enumerate}

\end{document}